\chapter{Équation maîtresse et états stationnaires}



\section{Discrétisation}

Nous commençons par proposer un modèle discret dans l'espace des charges et dans le temps décrivant la dynamique d'échange. Une limite vers le continu sera proposée dans le cadre du modèle linéaire.\\

En imposant une discrétisation de la charge, toute quantité devient un multiple d'une charge élémentaire $\Delta l$, l'ensemble des états occupables par une fourmi dans l'espace des charges est donc donné par $i*\Delta l$ où $i$ est un entier qui varie entre $0$ et $n$, $n$ vérifiant $n.\Delta l = 1$. Par la suite, nous noterons les états par leur indice; une fourmi dite dans l'état $j$ correspond à une fourmi transportant une quantité $j* \Delta l$.\\

\begin{figure}[h]
\centering
\includegraphics[scale=0.7]{Figures/Modele/discretisation.pdf}
\caption{Discrétisation de l'espace des charges. Chaque état occupable est caractérisé par un indice $i$ allant de $0$ à $n$.}
\label{discretisation}
\end{figure}


Le modèle présenté ici décrit la dynamique dans l'espace des charges générée par des échanges d'une gouttelette $\Delta l$ entre les individus à chaque pas de temps. La grandeur considérée sera donc $X(i,t)$: le nombre d'individus dans l'état $i$ au temps $t$, normalisée de sorte que:

\begin{equation}
\sum_i X(i,t)=1\\
\end{equation}

Cette normalisation nous encourage à redéfinir la quantité de nourriture ds la population:

\begin{equation}
Q_{tot}= \sum_i i X(i,t)\\
\end{equation}

\section{Taux de transition}
%Suite à une interaction,
%\footnote{Attention à l'ambiguïté entre les termes "interaction" et "échange". On parle d'interaction quand 2 individus se rencontrent et choisissent ou non d'échanger de la nourriture. On parle d'échange s'ils ont choisit d'échanger.}
Une fourmi dans l'état $j$ occupera après interaction avec une congénère l'état:
\begin{itemize}
\item[$\bullet$] $j-1$, si elle a donné
\item[$\bullet$] $j+1$, si elle a reçu
\item[$\bullet$] $j$, s'il n'y a pas eu d'échange.
\end{itemize}
Les autres états étant inaccessibles en un seul échange.Seules les transitions entre états voisins sont donc possibles entre 2 interactions. On note $W(i|j,t)$ la probabilité qu'un individu dans l'état $j$ au temps $t$ atteigne l'état $i$ au temps $t+1$, on trouve alors:


\begin{equation}
\begin{aligned}
W(i|j,t) = \delta_{j,i+1} W(i|i+1,t) + \delta_{j,i-1} W(i|i-1,t) + \delta_{i,j} W(i|i,t)\\
\label{Wij}
\end{aligned}
\end{equation}


Le premier terme du membre de droite donne la probabilité qu'un individu dans l'état $i+1$ se décharge d'une unité au cours de l'interaction. Soit la probabilité $P_D(i+1)$ que l'individu soit donneur multipliée par la probabilité qu'il interagisse avec un receveur: $\Sigma_R(t) = \sum_j X(j,t) P_R(j)$.\\
De même, le second terme est la probabilité $P_R(i+1)$ que l'individu soit receveur multipliée par la probabilité qu'il interagisse avec un donneur: $\Sigma_D(t) = \sum_j X(j,t) P_D(j)$.\\
Le troisième terme est la probabilité que l'individu n'échange pas: $W(i|i,t) = 1 - W(i-1|i,t)- W(i+1|i,t)$\\


En résumé:
\begin{equation}
\begin{aligned}
&W(i|i+1,t)= P_D(i+1) \Sigma_R(t)\\
&W(i|i-1,t)= P_R(i-1) \Sigma_D(t)\\
&W(i|i,t) = 1- W(i-1|i,t)- W(i+1|i,t)\\
\label{TransitionVoisins}
\end{aligned}
\end{equation}
où
\begin{equation}
\begin{aligned}
\Sigma_R	(t)	&= \sum_j P_R(j) X(j,t)\\
\Sigma_D(t) 	&= \sum_j P_D(j) X(j,t)\\
\label{SigmaRD}
\end{aligned}
\end{equation}



\section{Équation maîtresse}

La connaissance des taux de transitions entre états (équation \ref{Wij}) nous permet d'établir l'équation maîtresse pour $X(i,t)$. La forme la plus générale de l'équation maîtresse sur un espace discret \citep{vankampen} est donnée par:

\begin{equation}
\begin{aligned}
\dot{X}(i,t)= \sum_{j\neq i} W(i|j,t)X(j,t)-W(j|i,t)X(i,t)
\label{MasterGeneral}
\end{aligned}
\end{equation}

d'où

\begin{equation}
\begin{aligned}
\dot{X}(i,t)&= W(i|i+1) X(i+1,t) + W(i|i-1) X(i-1,t)\\
					&-W(i-1|i) X(i,t) -W(i+1|i) X(i,t)\\
		    &= P_D(i+1) \Sigma_R(t) X(i+1,t) + P_R(i-1) \Sigma_D(t) X(i-1,t)\\
					&-P_D(i) \Sigma_R(t) X(i,t) -P_R(i) \Sigma_D(t) X(i,t)\\
\label{Master}
\end{aligned}
\end{equation}
pour $i = 1,...,n-1$.\\

On fera attention aux bords $(i=0, i=n)$:
\begin{equation}
\begin{aligned}
\dot{X}(0,t) 	&= W(0|1) X(1,t) - W(1|0) X(0,t)\\
				&= P_D(1) \Sigma_R(t) X(1,t) - P_R(0) \Sigma_D(t) X(0,t)
\label{bord0}
\end{aligned}
\end{equation}
 et 
\begin{equation}
\begin{aligned}
\dot{X}(n,t) 	&= W(n|n-1) X(n-1,t) - W(n-1|n) X(n,t)\\
				&= P_R(n-1) \Sigma_D(t) X(n-1,t)- P_D(n) \Sigma_R(t) X(n,t)\\
\label{bordN}
\end{aligned}
\end{equation}


\section{Les états stationnaires}

\subsection{Forme générale pour la distribution stationnaire}
Recherchons la distribution stationnaire de l'équation maîtresse \ref{MasterGeneral}.\\

Pour les intervalles aux bords $(i=0,i=n)$, on reprend les équations \ref{bord0} et \ref{bordN}, pour les autres $(i=1,...,n-1)$ l'équation \ref{Master}.\\

$i=0$:
\begin{equation}
\begin{aligned}
0 &= W(0|1) X(1) - W(1|0) X(0)\\
  &= \Sigma_R P_D(1) X(1)- \Sigma_D P_R(0) X(0)\\
\label{statio_1}
\end{aligned}
\end{equation}

$i=n$:
\begin{equation}
\begin{aligned}
0 &= W(n|n-1) X(n-1) - W(n-1|n) X(n)\\
  &= \Sigma_D P_R(n-1)X(n-1)- \Sigma_R P_D(n) X(n)\\
 \label{statio_N}
\end{aligned}
\end{equation}

$i=1,...,n-1$:
\begin{equation}
\begin{aligned}
0 =& \Sigma_R P_D(i+1) X(i+1)+ \Sigma_D P_R(i-1) X(i-1)\\ 
	&- \Sigma_R P_D(i) X(i) - \Sigma_D P_R(i) X(i)\\
\label{statio_i}
\end{aligned}
\end{equation}

On peut alors résoudre ces équations "en cascade" et exprimer les X(i) ($i=1,...,n$) en fonction de X(0).\\
Avec l'équation \ref{statio_1}, on trouve facilement l'expression de X(1):

\begin{equation}
\begin{aligned}
X(1) = \frac{\Sigma_D}{\Sigma_R} \frac{P_R(0)}{P_D(1)} X(0)
\end{aligned}
\end{equation}

Et ensuite celle de X(2):
\begin{equation}
\begin{aligned}
X(2) = \frac{\Sigma^2_D}{\Sigma^2_R} \frac{P_R(0)P_R(1)}{P_D(1)P_D(2)} X(0)
\end{aligned}
\end{equation}

Par insertion dans les équations \ref{statio_i} et \ref{statio_N} on montre alors facilement que pour $i = 1,...,n$:
\begin{equation}
\begin{aligned}
X(i)&= X(i-1) \frac{\Sigma_D}{\Sigma_R} \frac{P_R(i-1)}{P_D(i)}  \\ 
	&= (\frac{\Sigma_D}{\Sigma_R})^i (\prod_{j=0}^{i-1} \frac{P_R(j)}{P_D(j+1)}) X(0).
\label{expression_Xi}
\end{aligned}
\end{equation}

Néanmoins $\frac{\Sigma_D}{\Sigma_R}$ dépend des valeurs des $X(i)$ (on rappelle les équations \ref{SigmaRD}) et $X(0)$ n'est pas encore déterminé, il manque donc 2 équations supplémentaires à notre système pour qu'il soit complet.\\

Ceci dit nous avons imposé la conservation du nombre d'individus et de la charge totale dans la population. Ces 2 relations de conservation vont nous fournir les 2 dernières équations nécessaires:
\begin{itemize}
	\item Conservation du nombre d'individus:
		\begin{equation}
		\begin{aligned}
		\sum_i X_i 	&= 1\\
					&= X_0+X_0 \sum_i^{N} (\frac{\Sigma_D}{\Sigma_R})^i \prod_{j=0}^{i-1}\frac{P_R(j)}{P_D(j+1)}
		\end{aligned}
		\end{equation}
		
	\item Conservation de la charge totale:
		\begin{equation}
		\begin{aligned}
		\sum_i i X_i 	&= Q_{tot}\\
					&= X_0 \sum_i^{N} i (\frac{\Sigma_D}{\Sigma_R})^i \prod_{j=0}^{i-1}\frac{P_R(j)}{P_D(j+1)}
		\end{aligned}
		\end{equation}
\end{itemize}

Nous arrivons donc à un système complet d'équations, qui nous permet d'exprimer univoquement la distribution $X(i)$. \\


Pour aller plus loin dans le calcul exact des distributions d'équilibre il est nécessaire de connaître explicitement les valeurs de $P_R$ et $P_D$. Nous l'effectuerons pour chacune des 4 lois étudiées.


\subsection{Loi constante}
Les probabilités $P_R$ et $P_D$ sont données par:

\begin{equation}
\left \{
\begin{aligned}
P_R(i) &= 0.5\\
P_D(i) &= 0.5
\end{aligned}
\right.
\end{equation}
pour $i=1,...,n-1$ avec les conditions aux bords \ref{ConditionsBordsProbabilite} pour i=0 et i=n.\\

La distribution stationnaire pour la loi constante vaut alors:

\begin{equation}
X_i=2 X_0 \alpha^i
\end{equation}
pour $i=1,...,n-1$ et 
\begin{equation}
X_n= X_0 \alpha^i
\end{equation}

avec $\alpha = \frac{1+X_n-X_0}{1-X_n+X_0}$.
On voit directement que cette distribution est une exponentielle de croissance $\alpha$. Nous avons toujours la loi de conservation du nombre d'individus:

\begin{equation}
\begin{aligned}
1&= \sum_i X_i\\
	&= X_0 (1+2\sum_{i=1}^{n-1}\alpha^i+\alpha^n)\\
	&= X_0 (2\sum_{i=0}^{n}\alpha^i-1-\alpha^n)\\
	&= X_0 (2 \frac{\alpha^{n+1}-1}{\alpha-1}-1-\alpha^n)\\
\end{aligned}
\end{equation}
donc
\begin{equation}
\begin{aligned}
X_0	&= \frac{\alpha-1}{\alpha^{n+1}-1+\alpha^n-\alpha}\\
	&= \frac{\alpha-1}{(\alpha^{n}-1)(\alpha+1)}\\
\end{aligned}
\end{equation}

Avec la conservation de la charge:

\begin{equation}
\begin{aligned}
Q	&= \sum_{i=0}^n i X_i\\
	&= \sum_{i=0}^{n-1} i X_i + n X_n\\
	&= 2 X_0 \sum_{i=0}^{n-1} i \alpha^i + n X_0 \alpha^n\\
	&= \frac{2\alpha -n\alpha^n-2\alpha^{n+1}+n\alpha^{n+2}}{(\alpha-1)^2}X_0\\
	&= \frac{2\alpha -n\alpha^n-2\alpha^{n+1}+n\alpha^{n+2}}{(\alpha^2-1)(\alpha^{n}-1)}\\
\end{aligned}
\end{equation}

Cette relation est très délicate à inverser, néanmoins nous pouvons obtenir numériquement la courbe d'$\alpha$ en fonction de $Q$ à $n$ fixé, et la représenter graphiquement à la figure \ref{AlphavsQCste}.\\

\begin{figure}[h]
\centering
\includegraphics[scale=0.65]{Figures/Maitresse/AlphavsQ.pdf}
\caption{Évolution du coefficient $\alpha$ en fonction de la quantité de nourriture dans la population. $n=100$}
\label{AlphavsQCste}
\end{figure}

A quantité de nourriture fixée, le coefficient $\alpha$ est univoquement déterminé. Une première interprétation de ce graphique est que la distribution de $X_i$ croit exponentiellement avec la charge pour une quantité de nourriture supérieure à $50\%$ de la capacité de la population et décroit pour une quantité inférieure à $50\%$.



\begin{figure}[h]
\centering
\includegraphics[scale=0.4]{Figures/Maitresse/Loi_cste.pdf}
\caption{Distribution stationnaire normalisée de la population dans l'espace des charges pour 6 quantités de nourriture différentes, dans le cadre du modèle ``constant''. Les quantités de nourriture correspondent à $10, 25, 35, 50, 65, 75\%$ de la capacité maximale de stock de la colonie. $n=100$}
\label{DistribAnalytiqueCste}
\end{figure}

On observe un étalement de la distribution sur l'espace des charges, moyenné par la quantité de nourriture disponible.

\subsection{Loi linéaire}
Les probabilités $P_R$ et $P_D$ sont données par:

\begin{equation}
\left \{
\begin{aligned}
P_R(i) &= \frac{n-i}{n}\\
P_D(i) &= \frac{i}{n}
\end{aligned}
\right.
\end{equation}

La distribution de l'état stationnaire est donnée par l'équation \ref{expression_Xi} et vaut alors pour la loi linéaire:

\begin{equation}
X_i=X_0 \alpha^i \prod_{j=0}^{i-1} \frac{n-j}{j+1} =X_0 \alpha^i \binom{n}{i}
\end{equation}

pour $i=1,...,n$. \\

$n$ étant le nombre de sites et $\alpha = \frac{\Sigma_D}{\Sigma_R} = \frac{Q}{nX_{tot}-Q}$.\\

On peut expliciter le nombre d'individus:
\begin{equation}
\begin{aligned}
X_{tot}=1&=\sum_{i=0}^{n}X_0\alpha^i\binom{n}{i}\\
		&= X_0 \sum_{i=0}^n1^{n-i}\alpha^i\binom{n}{i}\\
		&= X_0 (1+\alpha)^n 
\end{aligned}
\end{equation}
donc
\begin{equation}
X_0 = X_{tot}/(1+\alpha)^n=1/(1+\alpha)^n
\end{equation}

Du coup:

\begin{equation}
\begin{aligned}
X_i &= X_0 \alpha^i \binom{n}{i}\\
	&= \frac{\alpha^i}{(1+\alpha)^n} \binom{n}{i}\\
	&= \frac{1}{(1+\alpha)^{n-i}}\frac{\alpha^i}{(1+\alpha)^i}\binom{n}{i}
\end{aligned}
\end{equation}

pour $i=0,...,n$.

La distribution est donnée par une loi binomiale: $Binomiale(p=\frac{\alpha}{1+\alpha};n)$.

On montre en annexe que: $$\lim_{n\rightarrow\infty} Binomiale(p;n) = Normale(pn;np(1-p))$$

On peut calculer $np$ et $np(1-p)$, respectivement la moyenne et la variance de cette loi normale:
\begin{equation}
\begin{aligned}
np	&= n \frac{\alpha}{1+\alpha}\\
	&= n \frac{Q}{nX_{tot}}\\
	&= Q\\
\end{aligned}
\end{equation}

\begin{equation}
\begin{aligned}
np(1-p) &= Q(1-p)\\
		&= Q(1-\frac{Q}{n})\\
\end{aligned}
\end{equation}

La distribution de population dans l'espace des charges est donc une gaussienne centrée sur la charge moyenne. La largeur de la gaussienne est maximale quand les individus sont en moyenne à moitié chargés $(Q=n/2)$, et s'annule bien quand ceux-ci sont tous vides $(Q=0)$ ou tous remplis $(Q=n)$.


\begin{figure}[h]
\centering
\includegraphics[scale=0.4]{Figures/Maitresse/Loi_lineaire.pdf}
\caption{Distribution stationnaire de la population dans l'espace des charges pour 6 quantités de nourriture différentes, dans le cadre du modèle ``linéaire''. $n=100$}
\label{DistribAnalytiqueLin}
\end{figure}

On observe une concentration de la population autour de la charge moyenne par individu.



\subsection{Loi antilinéaire}
Les probabilités $P_R$ et $P_D$ sont données par:

\begin{equation}
\left \{
\begin{aligned}
P_R(i) &= \frac{i}{n}\\
P_D(i) &= \frac{n-i}{n}
\end{aligned}
\right.
\end{equation}

pour $i=1,...,n-1$ avec les conditions aux bords \ref{ConditionsBordsProbabilite} pour i=0 et i=n.\\

La distribution de l'état stationnaire est donnée par l'équation \ref{expression_Xi} et vaut alors pour la loi antilinéaire:

\begin{equation}
\begin{aligned}
X_i&=X_0 \alpha^i \frac{1\frac{1}{n}\frac{2}{n}...\frac{i-1}{n}}{\frac{n-1}{n}\frac{n-2}{n}...\frac{n-i}{n}}\\
&=X_0 \alpha^i \frac{n}{n-i}\frac{1}{\binom{n-1}{i-1}}
\end{aligned}
\end{equation}

pour $i=1,...,n-1$, et:
\begin{equation}
\begin{aligned}
X_n&=X_0 \alpha^n
\end{aligned}
\end{equation}

La complexité analytique de cette distribution nous empêche d'aller très loin dans l'analyse, nous recourons immédiatement à une analyse numérique pour retrouver la distribution de population stationnaire.


\begin{figure}[h]
\centering
\includegraphics[scale=0.4]{Figures/Maitresse/Loi_antilineaire.pdf}
\caption{Distribution stationnaire de la population dans l'espace des charges pour 6 quantités de nourriture différentes, dans le cadre du modèle ``antilinéaire''. $n=100$}
\label{DistribAnalytiqueAnti}
\end{figure}

On observe une distribution bimodale où les individus sont soient complètement chargés, soient vides. Ce modèle antilinéaire montre donc que la dynamique des échanges peut entraîner une séparation des rôles au sein d'une population d'individus identiques, seule une fraction de la population stocke la totalité de la nourriture.


\subsection{Loi vague}
Les probabilités $P_R$ et $P_D$ sont données par:

\begin{equation}
\left \{
\begin{aligned}
P_R(i) &= -2(\frac{i}{n})^2+1.5\frac{i}{n}+0.5\\
P_D(i) &= 2(\frac{i}{n})^2-1.5\frac{i}{n}+0.5
\end{aligned}
\right.
\end{equation}

pour $i=1,...,n-1$ avec les conditions aux bords \ref{ConditionsBordsProbabilite} pour i=0 et i=n.\\

La distribution de l'état stationnaire donnée par l'équation \ref{expression_Xi} et les lois de conservation est inexploitable analytiquement. On recourt immédiatement à une analyse numérique pour retrouver la distribution stationnaire.


\begin{figure}[h]
\centering
\includegraphics[scale=0.4]{Figures/Maitresse/Loi_vague.pdf}
\caption{Distribution stationnaire de la population dans l'espace des charges pour 6 quantités de nourriture différentes, dans le cadre du modèle ``vague''. $N=10000; n=100$}
\label{DistribAnalytiqueVague}
\end{figure}

On observe à nouveau une distribution bimodale pour de faibles quantités de nourriture, et donc à nouveau une séparation des rôles engendrée par la dynamique d'échange. Pour de plus grandes quantité de nourriture la bimodalité disparaît et la distribution se concentre autour de la charge moyenne par individu.