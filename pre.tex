%% packages


%\usepackage[american]{babel}
%\usepackage[latin1]{inputenc}
%\usepackage[left=2.5cm,top=2cm,right=2.5cm,nohead,nofoot]{geometry}
%\usepackage{url}




%\usepackage{a4wide} %smaller margins
\usepackage[comma,sort&compress]{natbib}



% pour des marges plus adaptées au format A4
\usepackage{a4}

% pour pouvoir écrire des caractères accentués, entre autres
\usepackage{ae}
%\usepackage{aeguill}

% pour rajouter automatiquement un espace après certaines commandes du
% package babel avec l'option français (ex. : \ieme)
%\usepackage{xspace}

% pour utiliser des environnements mathématiques évolués
%\usepackage{amsmath}

% pour avoir accès à certains symboles mathématiques
%\usepackage{amssymb}

% pour pouvoir inclure des figures au format EPS (entre autres)
%\usepackage{graphicx}


%% commandes

% supprime l'indentation de la première ligne d'un paragraphe
%\setlength{\parindent}{0em}

% rajoute un espace entre paragraphes
%\setlength{\parskip}{0.5em}









%\usepackage{a4wide} %smaller margins
%\usepackage[comma,sort&compress]{natbib}
%\usepackage{hyperref}
\usepackage[utf8]{inputenc}
\usepackage[T1]{fontenc} % copiable text (accents/special chars) on the pdf
%\usepackage{lmodern} % better font
%\usepackage[style=french]{csquotes}
\usepackage[french]{babel} % french typography, hyphenation, captions
%\uchyph=0 % TeX command to disable hyphenation on uppercase words
%\usepackage{ifpdf} % pdflatex check
\usepackage{amsmath}  %
\usepackage{amsfonts} %  math stuff
\usepackage{amssymb}  %
%\usepackage{nicefrac} %
%\usepackage{setspace} %for mathematica

% Tables and floats
%\usepackage[labelfont=bf,textfont={small,it},justification=centering]{caption} % change the caption style to something more discreet
%\setlength{\doublerulesep}{\arrayrulewidth} % double \hline or vertical rule gives a thick line
%\usepackage{multirow} % row spanning on tables
\usepackage{float}    % provides strict H position for floats
\usepackage{graphicx}
\usepackage{subfig}
\usepackage{psfrag}

\usepackage{verbatim} % comment environment

%\usepackage[bottom]{footmisc} % place footnotes at the bottom of the page in all cases

\usepackage{xspace} %to use at the end of inline text macros for sensible spacing

%document-specific stuff
\usepackage{lscape} %landscape environment
\usepackage{caption}

\usepackage{color}


\usepackage{algorithm}  %
\usepackage{algorithmicx}  %
\usepackage{algpseudocode}

