\chapter{Simulations stochastiques sur une population homogène}
\section{Simulations stochastiques}
En plus de l'approche analytique jusqu'ici considérée, nous pouvons étudier la dynamique sur base de simulations stochastiques. Cette approche nous permettra d'étudier de nouveaux aspects relevants pour la dynamique, où l'équation maîtresse néglige une partie de l'information sur le caractère stochastique des échanges.\\

Dans un premier temps ces simulations serviront de contrôle pour nos calculs théoriques effectués au chapitre précédent, on s'assurera que les distributions d'équilibre atteintes par simulation soient bien identiques aux distributions calculées analytiquement. Dans un second temps, ces simulations nous permettront d'obtenir de nouveaux résultats difficiles, voire impossibles à retrouver analytiquement, comme l'étude de l'ergodicité du système (présentée plus bas) ou la détermination d'un temps caractéristique de retour à l'équilibre.\\

Les simulations introduites ici considèrent $N$ individus chargés, qui tenteront des échanges d'une unité de charge avec un partenaire choisi aléatoirement. Bien entendu, les probabilités d'échange sont définies comme au chapitre 2 et la quantité de nourriture répartie dans la population, comme le nombre d'individus, sont des grandeurs conservées pendant la simulation.\\

Nous considérons 2 simulations différentes:
\begin{itemize}
\item[$\bullet$] des simulations à échanges synchrones, où à chaque pas de temps $\frac{N}{2}$ couples sont formés et chacun tentera un échange
\item[$\bullet$] des simulations asynchrones, où à chaque pas de temps 1 seul couple est formé et tentera d'échanger.\\
\end{itemize}

Nous montrerons que ces simulations sont identiques moyennant un facteur de conversion des échelles de temps et choisirons d'utiliser uniquement les simulations synchrones.\\

Les algorithmes des 2 simulations sont présentés en pseudo code en annexe de ce mémoire.


\subsection{Équivalence des simulations}

Comme annoncé, nous montrons ici que les simulations synchrones et asynchrones sont identiques moyennant un facteur de conversion des échelles temporelles, et représentent donc la même dynamique. Chaque pas de temps de la simulation synchrone génère $\frac{N}{2}$ interactions, contre $1$ pour la simulation asynchrone; on s'attend donc à un facteur de conversion $\frac{N}{2}$.\\

Pour quantifier l'évolution temporelle du système via les simulations, nous mesurons l'évolution du moment d'ordre 2 de la distribution de population dans l'espace des charges. Les moments d'ordre $0$ et $1$ correspondent respectivement au nombre d'individus et à la quantité de nourriture dans la population (équation \ref{moment01}), qui sont 2 grandeurs invariantes et ne peuvent nous donner d'informations sur l'évolution temporelle du système.

\begin{equation}
\begin{aligned}
<l^0> &= \int_0^1 X(l,t) dl = 1\\
<l^1> &= \int_0^1 l X(l,t) dl = Q\\
\end{aligned}
\label{moment01}
\end{equation}

Le moment d'ordre 2 est donc donné par:

\begin{equation}
\begin{aligned}
<l^2>(t) &= \int_0^1 l^2 X(l,t) dl \\
\end{aligned}
\label{moment2continu}
\end{equation}


Nous comparons le moment d'ordre 2 mesuré sur chacune des simulations pour 2 lois différentes et, à chaque fois, 2 quantités de nourriture différentes; les autres paramètres de simulation sont laissés identiques. Les lois considérées sont la loi constante et la loi antilinéaire, et nous prenons des quantités de nourriture correspondantes à $20$ et $50\%$ de la capacité maximale de la population. Les résultats sont représentés aux figures \ref{EquivalenceCste} et \ref{EquivalenceAnti}.


\begin{figure}[h]
\centering
\includegraphics[scale=0.35]{Figures/Homogene/crop_M2_cste_Q2050.pdf}
\caption{Comparaison des simulations synchrones et asynchrones dans le cadre de la loi constante, pour des quantités de nourriture correspondantes à $20\%$ (à gauche) et $50\%$(à droite) de la capacité maximale de la population. On a converti l'échelle de temps des simulations synchrones d'un facteur $\frac{N}{2}$. Paramètres: $N=500$ individus, $n = 100$. Courbes moyennées sur 20 simulations.}
\label{EquivalenceCste}
\end{figure}

\begin{figure}[h]
\centering
\includegraphics[scale=0.35]{Figures/Homogene/crop_M2_anti_Q2050.pdf}
\caption{Comparaison des simulations synchrones et asynchrones dans le cadre de la loi antilinéaire, pour des quantités de nourriture correspondantes à $20\%$ (à gauche) et $50\%$(à droite) de la capacité maximale de la population. On a converti l'échelle de temps des simulations synchrones d'un facteur $\frac{N}{2}$. Paramètres: $N=500$ individus, $n = 100$. Courbes moyennées sur 20 simulations.}
\label{EquivalenceAnti}
\end{figure}


Dans tous les cas, les courbes obtenues via les simulations synchrones et celles via les simulations asynchrones se superposent; les simulations synchrones et asynchrones représentent bien la même dynamique. Notons bien que nous n'avons représenté ici qu'un échantillon des cas de figures (2 lois et 2 quantités de nourriture différentes), les autres cas étant équivalents nous avons jugé inutile de représenter une multitude de figures pour appuyer notre argumentation.

\section{Comparaison modèle - simulations}

Maintenant que nous savons que les simulations synchrones et asynchrones sont équivalentes, nous pouvons nous contenter d'utiliser les simulations synchrones pour la suite du travail. \\

Le premier intérêt de nos simulations est de s'en servir comme contrôle du modèle analytique proposé au chapitre précédent. Nous allons donc nous assurer que les distributions stationnaires obtenues via les simulations correspondent à celles calculées analytiquement sur base du modèle au chapitre 5. La comparaison sera faite pour chaque loi, et à chaque fois pour 6 quantités de nourriture différentes. Les résultats sont repris aux figures \ref{ComparaisonMScste}, \ref{ComparaisonMSlineaire}, \ref{ComparaisonMSanti} et \ref{ComparaisonMSvague}.

\begin{figure}[h]
\centering
\includegraphics[scale=0.35]{Figures/Homogene/crop_ComparaisonMScste.pdf}
\caption{Comparaison des distributions stationnaires obtenues via les simulations aux distributions analytiques du modèle, dans le cadre d'une loi constante pour 6 quantités de nourriture différentes. Paramètres de simulations: $10000$ individus, $n=100$. Courbes moyennées sur 20 simulations.}
\label{ComparaisonMScste}
\end{figure}

\begin{figure}[h]
\centering
\includegraphics[scale=0.35]{Figures/Homogene/crop_ComparaisonMSlineaire.pdf}
\caption{Comparaison des distributions stationnaires obtenues via les simulations aux distributions analytiques du modèle, dans le cadre d'une loi linéaire pour 6 quantités de nourriture différentes. Paramètres de simulations: $10000$ individus, $n=100$. Courbes moyennées sur 20 simulations.}
\label{ComparaisonMSlineaire}
\end{figure}

\begin{figure}[h]
\centering
\includegraphics[scale=0.35]{Figures/Homogene/crop_ComparaisonMSanti.pdf}
\caption{Comparaison des distributions stationnaires obtenues via les simulations aux distributions analytiques du modèle, dans le cadre d'une loi antilinéaire pour 6 quantités de nourriture différentes. Paramètres de simulations: $10000$ individus, $n=100$. Courbes moyennées sur 20 simulations.}
\label{ComparaisonMSanti}
\end{figure}

\begin{figure}[h]
\centering
\includegraphics[scale=0.35]{Figures/Homogene/crop_ComparaisonMSvague.pdf}
\caption{Comparaison des distributions stationnaires obtenues via les simulations aux distributions analytiques du modèle, dans le cadre d'une loi vague pour 6 quantités de nourriture différentes. Paramètres de simulations: $10000$ individus, $n=100$. Courbes moyennées sur 20 simulations.}
\label{ComparaisonMSvague}
\end{figure}

\section{Différentiation des individus à l'équilibre}

Dans cette section, nous recourons aux simulations pour répondre à une question simple: une fois à l'équilibre, sur quelle échelle de temps les individus sont-ils différentiables ou combien de temps faut-il pour qu'une fourmis fortement chargée devienne se vide et inversement, une fourmis presque vide devienne très chargée?\\

Cette question est délicate, comment définir précisément un individu ``très chargé'' ou un individu ``quasiment vide''? Par exemple dans le modèle à loi linéaire, les individus se concentrent tous autour d'une charge moyenne, alors que dans le modèle antilinéaire les individus seront complètement remplis ou complètement vides.\\

Néanmoins grâce aux simulations nous pouvons suivre la trajectoire à l'équilibre d'un seul individu dans l'espace des charges et mesurer la fréquence d'occupation de chaque état durant un laps de temps fixé. Cette mesure peut être effectuée sur toute la population, il est donc possible de comparer la fréquence d'occupation de chaque état d'un individu à l'autre.\\


On définit alors sur base des simulations:

\begin{itemize}

\item[$\bullet$] $q_f[t]$\\
L'état de la fourmi $f$ au pas de temps $t$. ($f=1,...,10000$)\\

\item[$\bullet$] $h_{f,\tau}(i)=\frac{1}{\tau} \sum_{t=1}^{\tau}\delta_{q_f[t]}^i$ \\
La fréquence d'occupation de l'état $i$ par la fourmis $f$ après $\tau$ interactions, simulées à l'équilibre.\\


\item[$\bullet$] $\mu_{\tau}(i) = \frac{1}{N}\sum_{f=1}^N h_{f,\tau}(i)$ \\
La moyenne des fréquences d'occupation de l'état $i$ sur toutes les fourmis.\\

\item[$\bullet$] $\sigma_{\tau}(i) = \sqrt{\frac{1}{N-1}\sum_{f=1}^N (h_{f,\tau}(i)-\mu_{\tau}(i))^2}$ \\
L'écart-type des fréquences d'occupation de l'état $i$ sur toutes les fourmis.\\
\end{itemize}

Une première observation est que la courbe $\mu_{\tau}(i)$ correspond à la distribution stationnaire $X(i)_{statio}$ quelque soit la période $\tau$ choisie. En effet:

\begin{equation}
\begin{aligned}
\mu_{\tau}(i) 	&= \frac{1}{N}\sum_{f=1}^N h_{f,\tau}(i)\\
				&= \frac{1}{N}\sum_{f=1}^N \frac{1}{\tau} \sum_{t=1}^{\tau}\delta_{q_f[t]}^i\\
				&= \frac{1}{\tau} \sum_{t=1}^{\tau} \frac{1}{N}\sum_{f=1}^N \delta_{q_f[t]}^i\\
				&= \frac{1}{\tau} \sum_{t=1}^{\tau} X(i)_{statio}\\
				&= X(i)_{statio}
\end{aligned}
\end{equation}

La grandeur relevante ici est l'écart-type des fréquences d'occupation des états, pris sur la population. Cette grandeur caractérise la différence entre les individus, elle atteint 0 quand toutes les fourmis ont occupé chaque état avec la même fréquence et est d'autant plus grande que les fourmis ont tendance à ``stagner'' dans leur état.  \\

Nous nous intéressons donc à la dépendance en $\tau$ de cette grandeur. Nous illustrons dans un premier temps cette dépendance en observant la variation de l'écart-type pour 3 valeurs de $\tau: 500, 1000$ et $5000$. On porte donc en graphique l'écart à la distribution moyenne (i.e. stationnaire) dans l'espace des charges pour 2 lois différentes (constante et vague), voir figures \ref{ergo_vague} et \ref{ergo_cste}. Comme attendu l'écart-type diminue quand $\tau$ augmente; en effet, au plus le nombre d'interactions augmente, au plus chaque individu va se déplacer dans l'espace des charges, et les fréquences d'occupation des états s'uniformiseront sur la population. On remarque aussi que contrairement à la loi constante, l'écart-type moyen de la loi vague reste élevé, ce qui montre que sur les temps considérés les individus ont tendance à stagner à proximité de leur état initial, ceci exprime la difficulté que rencontrent les individus peu chargé à ``rejoindre' la sous-population chargée.  \\



\begin{figure}[h]
\centering
\includegraphics[scale=0.35]{Figures/Homogene/crop_ergo_vague.pdf}
\caption{Représentation de $\mu_{\tau}$ et $\mu_{\tau}\pm \sigma_{\tau}$ en fonction de la charge $i$. Paramètres: Loi vague, $Q=25$, $250$ individus, $n=100$.}
\label{ergo_vague}
\end{figure}

\begin{figure}[h]
\centering
\includegraphics[scale=0.25]{Figures/Homogene/crop_ergo_cste.pdf}
\caption{Représentation de $\mu_{\tau}$ et $\mu_{\tau}\pm \sigma_{\tau}$ en fonction de la charge $i$. Paramètres: Loi constante, $Q=25$, $250$ individus, $n=100$.}
\label{ergo_cste}
\end{figure}


Nous pouvons mesurer précisément la dépendance de $\sigma_{\tau}(i)$ en $\tau$ pour chaque loi et différentes quantités de nourriture. Ceci dit $\sigma$ dépend explicitement de l'état $i$, afin d'avoir un point de vue plus globale sur la dépendance temporelle de l'écart-type $\sigma$ nous choisissons de moyenner cette grandeur sur l'ensemble des états et de considérer alors:

\begin{equation}
<\sigma>(\tau)=\frac{1}{n}\sum_i^n \sigma_{\tau}(i)
\label{moySTD}
\end{equation}

Les résultats sont repris aux figures \ref{ergo_cste_STD}, \ref{ergo_lineaire_STD}, \ref{ergo_anti_STD} et \ref{ergo_vague_STD} pour chacune des 4 lois considérées et à chaque fois 6 quantités de nourriture différentes.\\


Une première analyse de ces courbes montre qu'elles ne peuvent être approchées par un ajustement exponentiel ou en loi de puissances. L'espoir de les caractériser réside en une analyse plus approfondie tenant compte des possibilités de superposition d'une loi de puissance avec une loi exponentielle:

\begin{equation}
<\sigma>(\tau)=A exp(\frac{\tau}{\tau^*}) + \sum_k a_k \tau^k
\end{equation}

L'intérêt étant alors de pouvoir exprimer un temps caractéristique de décroissance de $<\sigma>$ qui quantifierait le temps qu'une fourmis prend pour se déplacer dans l'espace des charges .\\

\begin{figure}[h]
\centering
\includegraphics[scale=0.4]{Figures/Homogene/crop_STDvsTau_cste.pdf}
\caption{Évolution de l'écart-type moyen en fonction du nombre d'échanges (équation \ref{moySTD}). Paramètres: Loi constante, $100$ individus, $n=100$.}
\label{ergo_cste_STD}
\end{figure}

\begin{figure}[h]
\centering
\includegraphics[scale=0.4]{Figures/Homogene/crop_STDvsTau_lineaire.pdf}
\caption{Évolution de l'écart-type moyen en fonction du nombre d'échanges (équation \ref{moySTD}). Paramètres: Loi linéaire, $100$ individus, $n=100$.}
\label{ergo_lineaire_STD}
\end{figure}

\begin{figure}[h]
\centering
\includegraphics[scale=0.4]{Figures/Homogene/crop_STDvsTau_anti.pdf}
\caption{Évolution de l'écart-type moyen en fonction du nombre d'échanges (équation \ref{moySTD}). Paramètres: Loi antilinéaire, $100$ individus, $n=100$.}
\label{ergo_anti_STD}
\end{figure}

\begin{figure}[h]
\centering
\includegraphics[scale=0.4]{Figures/Homogene/crop_STDvsTau_vague.pdf}
\caption{Évolution de l'écart-type moyen en fonction du nombre d'échanges (équation \ref{moySTD}). Paramètres: Loi vague, $100$ individus, $n=100$.}
\label{ergo_vague_STD}
\end{figure}

\clearpage
\section{Temps caractéristique d'arrivée à l'équilibre}

Nous nous sommes jusqu'ici concentrés sur la dynamique d'équilibre, retrouvant la distribution de population stationnaire ou cherchant à caractériser le temps de passage d'un état riche à un état pauvre. Dans cette section nous nous intéressons à la dynamique hors équilibre, en particulier on cherchera à déterminer un temps caractéristique d'arrivée à l'état d'équilibre.\\

Bien que nous ayons montré que les distributions stationnaires sont indépendantes des conditions initiales (i.e. de la répartition de la nourriture au sein du groupe), rien ne dit qu'il en soit de même pour la vitesse d'arrivée à cette distribution. Dès lors nous nous intéressons à l'évolution des distributions de population pour différentes répartitions initiales d'une même quantité de nourriture.\\

Afin de quantifier l'état d'une population et d'étudier son évolution temporelle, nous reprenons le second moment de la distribution (équation \ref{moment2}), déjà introduit à la section 4.1. Nous comparons donc l'évolution de ce second moment en fonction du temps pour 6 conditions initiales différentes. 

\begin{equation}
\begin{aligned}
<l^2>(t) &= \int_0^1 l^2 X(l,t) dl 
\end{aligned}
\label{moment2}
\end{equation}


Les résultats sont résumés ci-dessous. Une première analyse semble montrer une croissance exponentielle de $<l^2>$ vers sa valeur stationnaire, on peut donc écrire $<l^2>(t)$ comme:

\begin{equation}
\begin{aligned}
<l^2>(t)=A e^{-\frac{t}{\tau}}+<l^2>_{statio}
\end{aligned}
\end{equation}

où

\begin{equation}
\begin{aligned}
<l^2>_{statio}= \int_0^1 l^2 X(l)_{statio} dl
\end{aligned}
\end{equation}

et $\tau$ est le temps caractéristique d'arrivée à l'équilibre. Ceci dit une analyse plus fine sera nécessaire pour déterminer dans chaque situation (loi et quantité de nourriture fixées) le temps caractéristique d'arrivée à l'équilibre qui dépend vraisemblablement de la condition initiale.