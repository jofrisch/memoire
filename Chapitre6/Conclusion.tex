\chapter{Conclusion}

\section{Résultats et discussions}

%La compréhension de l'organisation d'une fourmilière intéresse la communauté scientifique depuis plusieurs décennies.

Dans ce mémoire nous avons construit un nouveau modèle décrivant la dynamique de distribution de nourriture au sein d'une colonie de fourmis. Ce sont les interactions sociales entre les individus qui sont à l'origine des flux et de la distribution de nourriture à l'échelle de la population. Le point central de la modélisation est consacré à l'influence des comportements d'échange entre individus sur la répartition globale de la nourriture. Pour concevoir et analyser ce modèle nous avons utilisé des outils classiques issus de la mécanique statistique et plus exactement de l'étude des processus stochastiques, comme les équations maîtresses présentées au chapitre 3 ou les simulations stochastiques d'échanges entre les individus aux chapitres 4 et 5. \\

Par manque de données expérimentales quantitatives, nous avons choisi d'étudier en particulier 4 lois comportementales simples et extrêmes, compatibles avec les comportements observés ou supposés des fourmis, et qualitativement différentes. Pour chaque loi nous avons pu caractériser la répartition des stocks de nourriture dans la colonie lorsque l'état stationnaire est atteint, ce qui correspond en pratique à des temps de l'ordre de quelques heures.\\

Comme nous l'avons déjà signalé, la distribution des réserves varie fortement entre les individus d'une même colonie. Deux hypothèses extrêmes sont discutées dans la littérature expérimentale \citep{Saragosti_2013}: ces distributions sont le résultat de différences physiologiques ou comportementales entre individus ou bien ces distributions résultent de la dynamique des échanges. A la lumière d'un ensemble de résultats, il est hautement probable que ces deux hypothèses ne sont pas exclusives. Cependant la modélisation nous permet d'explorer des situations ``caricaturales'' où une seule des hypothèses est testée.\\

Dans le cas d'une population constituée d'individus identiques (population homogène), nous avons montré que la distribution de nourriture entre les individus est alors particulièrement sensible au comportement des individus.  Dans le cas où les règles d'échanges favorisent une distribution homogène des réserves entre individus (loi constante et linéaire), une distribution exponentielle ou gaussienne de charge caractérise l'état stationnaire. Mais dans le cas de loi antilinéaire et la loi vague la dynamique d'échange entraîne une séparation des rôles, une différentiation  entre les individus. On remarque que certains serviront de ``dépôt'' chez lesquels les autres viendront ``se vider''. Cette séparation des rôles s'observe dans des populations parfaitement homogènes. Ce résultat nous semble important car il montre qu'il n'est pas nécessaire, contrairement à l'hypothèse classique, que les individus présentent des caractéristiques intrinsèques (i.e. morphologiques) différentes pour observer des différences entre les individus au niveau du stockage. Ce résultat peut se comprendre en ce sens que ces deux lois d'échange conduisent à des boucles positives, à des amplifications qui favorisent la différentiation.\\

Notre travail s'est également porté sur l'étude de populations hétérogènes, i.e. d'individus non-identiques, le but étant de mettre en évidence le rôle occupé par un individu en fonction de ses caractéristiques morphologiques. Nous avons donc reformulé le modèle pour établir les premières différences entre les populations homogènes et hétérogènes. Nous nous sommes limités dans ce travail à une situation où d'une part les différences entre individus étaient restreintes à leur capacité de stockage et d'autre part à un nombre égal de fourmis à forte et faible capacité de stockage. Avec les résultats actuellement obtenus, aucune relation définitive n'a été établie entre les caractéristiques d'une fourmi et son rôle dans la colonie. Cependant il apparaît qu'avec les deux premières lois (linéaire et constante), la proportion de nourriture stockée par chacune des sous-populations est proche de la proportionnalité de sa capacité de stockage. Dans le cas des deux autres lois (antilinéaire et vague), d'une part nous ne retrouvons pas systématiquement cette proportionnalité et d'autre part la distribution initiale des réserves influence la distribution stationnaire. En particulier pour la loi antilinéaire, pour certaines distributions initiales ce sont uniquement les individus à faible capacité qui accumuleront la majorité des réserves, alors que pour d'autres distributions initiales seuls ceux à forte capacité s'accapareront des réserves. Ces résultats à la fois non intuitifs et contradictoires avec certaines données expérimentales devront être approfondis dans le futur.\\

%Nous avons montré que la distribution de nourriture est alors particulièrement sensible au comportement des individus: dans certain cas la dynamique d'échange entraîne une séparation des rôles entre les domestiques, on remarque par exemple que certaines serviront de ``dépôt'' chez lesquels les autres viendront ``se vider''. Cette séparation des rôles s'observe dans des populations parfaitement homogènes, ce qui contredit l'hypothèse - classique - que le rôle d'une fourmi au sein de la colonie est uniquement déterminé par ses caractéristiques intrinsèques (i.e. morphologiques), celui-ci pouvant donc être défini par la dynamique d'échanges entre les individus.\\ 
%
%Notre travail s'est également porté sur l'étude de populations hétérogènes, i.e. d'individus non-identiques, le but étant de mettre en évidence le rôle occupé par un individu en fonction de ses caractéristiques morphologiques. Nous avons pu reformuler le modèle et établir les premières différences entre les populations homogènes et hétérogènes. Cette partie du travail n'est pas terminée et de nombreux points sont à préciser. Ceci dit, avec les résultats actuellement obtenus, aucune relation évidente n'a encore été établie entre les caractéristiques d'une fourmi et son rôle dans la colonie.\\


\section{Perspectives du travail et applications}

Ce mémoire ouvre différentes perspectives d'enrichissement du modèle proposé. Par définition, aucun modèle n'est exact et chacune de nos hypothèses peut faire l'objet d'une discussion critique. En particulier, on peut s'intéresser à celle formulée par l'équation \ref{PRPD1}: un individu se place obligatoirement comme receveur ou (exclusif!) donneur durant un échange. Cette hypothèse interdit ``l'indifférence'' chez les individus, elle implique par exemple qu'une fourmi pleine tentera toujours de donner de la nourriture puisqu'elle ne peut en recevoir, chose discutable! Reconsidérer cette hypothèse permettrait éventuellement de s'affranchir des discontinuités observées aux bords de certaines distributions, mais aussi de voir comment cette indifférence peut modifier la distribution des réserves.\\

Au niveau des populations présentant des différences intrinsèques, nous nous sommes limités à l'étude d'un seul paramètre: la capacité maximale de stockage. Il est évident que les autres paramètres doivent être étudiés et en particulier les différences entre individus au niveau des probabilités intrinsèques d'être donneur ou receveur.\\

Nous avons aussi modélisé des échanges d'une quantité fixe de nourriture au cours des trophallaxies, ce qui est sans doute une simplification forte. Différentes données expérimentales suggèrent qu'il y ait une variabilité des quantités de nourriture transférées lors d'un échange \citep{Cassill_1999}. Une amélioration du modèle serait donc de redéfinir, sur base de données expérimentales, la quantité échangée - via une distribution $P(x)$ exprimant la probabilité qu'une quantité $x$ soit échangée au cours d'une trophallaxie - et d'étudier l'impact de cette modification sur la répartition globale de la nourriture.\\

Il nous semble particulièrement important de tenir compte d'une dynamique spatiale où les individus se déplacent au sein du nid. Ceci permettrait un choix plus ``naturel'' des paires effectuant des échanges: les paires seraient formées par les fourmis proches les unes des autres. Les expériences montrent clairement l'agrégation des stocks à l'intérieur du nid \citep{buffin_feeding_2009}. Il est naturel de se poser la question si cette prise en compte de l'espace, et la contrainte en résultant sur les échanges, pourrait à elle seule modifier la dynamique de répartition de la nourriture 	dans la population, mais également reproduire l'agrégation spatiale des réserves observée expérimentalement.\\

Le modèle incluant l'espace pourrait alors être comparé à ces résultats expérimentaux, ce qui permettrait d'effectuer un premier tri parmi les différentes hypothèses concernant les échanges ou l'hétérogénéité des différences intrinsèques des insectes. \\


Concernant la validation expérimentale des règles d'échange, le modèle est ``demandeur'' d'expérimentations et de quantifications. Nous pensons que des méthodes basées sur la scintigraphie et de la nourriture marquée pourraient répondre en partie à ces questions. Des expériences ``préliminaires'' anciennes avaient été menées mais à notre connaissance il n'y avait pas eu de suite \citep{bonavita-cougourdan_nouvelle_1979}. La procédure viserait à identifier les flux d'échange entre deux individus en fonction de la charge des fourmis.\\



%En plus de cette contrainte sur les échanges qui pourrait à elle seule modifier la dynamique de répartition de la nourriture dans la population, cette considération permettrait également d'étudier la distribution spatiale de la nourriture. Il s'avère que des études expérimentales ont déjà été menées en ce sens \citep{buffin_feeding_2009}, une comparaison entre ces résultats et les prévisions du modèle permettrait éventuellement de confirmer ce dernier.\\
%

Enfin il est tentant de construire un modèle plus général décrivant complètement la dynamique de gestion de la nourriture au sein d'une fourmilière. Pour cela il est nécessaire de coupler notre travail avec un modèle de récolte de nourriture. Ce couplage ne semble pas présenter de difficulté majeure, en effet de nombreux modèles et expériences ont été menés concernant ce point \citep{sumpter_collective_2010}. Les enjeux biologiques sont alors différents: il est nécessaire d'inclure les processus de recherche de nourriture et de recrutement. Le couplage imposerait des modifications au présent modèle en incluant des nouveaux états aux individus selon que ceux-ci se situent à l'intérieur ou à l'extérieur du nid.\\

Une question qui se pose est notamment de voir comment le flux de nourriture récoltée influence la formation et l'organisation des réserves ou comment les échanges influencent la récolte \citep{Provecho_2009}. Ce modèle plus complet pourrait alors être comparé à la dynamique globale expérimentale de formation des réserves: en effet expérimentalement l'apport de nourriture se fait en parallèle de l'organisation des réserves, bien que celle-ci se fasse plus lentement. Cette différence de temps justifiait notre choix de modélisation de considérer en premier lieu rien que la distribution de la nourriture entre les individus.\\

% Un tel modèle serait donc à construire depuis 0, comme nous l'avons fait pour la dynamique d'échange. Le couplage imposerait des modifications au présent modèle, il pourrait se faire via une reformulation en terme d'un modèle d'individus, en ajoutant un état ``in'' ou ``out'' aux fourmis, selon qu'elles soient dans le nid et puissent échanger ou non. Une alternative est la définition d'un système ouvert, N varie au cours du temps, où les fourmis peuvent rentrer et sortir de la fourmilière en transportant une quantité de nourriture. Ceci dit, un modèle de récolte a déjà été proposé par des physiciens \citep{William2012} et le couplage avec notre modèle ne semble pas présenter de difficultés majeures. De bonne augure...\\

 
Outre l'étude de la gestion de nourriture dans une fourmilière, le modèle présenté dans ce mémoire trouvera son intérêt à de multiples applications, parfois très éloignées de la biologie théorique. En économie par exemple des modèles étudient les échanges d'argent entre traders, considérant $N$ individus effectuant des transactions financières, dans un circuit fermé \citep{lopez-ruiz_complex_2012}. On comprend immédiatement le lien avec notre travail, où $N$ fourmis s'échangent une quantité fixe de nourriture. Ceci dit quelques différences persisteront toujours entre les fourmis et la société humaine, les premières étant limitées par la quantité transportable alors qu'un ``bon'' trader pourra posséder la totalité des biens disponibles. En robotique, on pourra citer le projet européen SYMBRION (symbrion.org) qui consiste à faire échanger de la matière et/ou de l'énergie entre des robots dans un but préalablement défini: déblayer un terrain, explorer un milieu inconnu, transporter de l'énergie d'un point A à un point B, etc. Comme dans notre travail, c'est le comportement individuel régissant les interactions entre robots qui déterminera l'efficacité de la population. Notre modèle peut donc servir à définir les lois d'échanges entre les robots qui permettent d'optimiser leur rendement.
