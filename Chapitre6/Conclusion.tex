\chapter{Conclusion}

\section{Résultats et discussions}
1. On a construit un nouveau modèle décrivant la dynamique d'échange au sein d'une fourmilière, mettant en évidence l'influence du comportement individuel sur l'organisation générale de la population.
2. On a utilisé des outils classiques de mécanique statistique et des simulations stochastiques pour étudier quantitativement ce modèle.
3. On a pu caractériser la répartition des stocks de nourriture à l'échelle de la population pour différentes lois comportementales de l'individu.
4. On a montré qu'une séparation des rôles au sein de la population peut être générée uniquement par la dynamique d'échange et pas obligatoirement par des prédispositions morphologiques.
5. On s'est intéressé à l'influence de caractères hétérogènes dans la population en reformulant le modèle initial. 
6. On a pu caractériser l'influence du caractère hétérogène de la population sur la répartition des de la nourriture pour les lois comportementales étudiées.




\section{Proposition de protocole expérimental}
Comment vérifier le modèle, à discuter avec JL.

\section{Perspectives du travail et applications}
A moyen terme:
Plusieurs hypothèses ont été faites dans le cadre de ce travail
- $P_R(l) + P_D(l) =1$? => discontinuité aux bords
- des échanges de plus d'une gouttelette

- paramétriser les lois comportementales

A long terme:
- étudier une dynamique spatiale
- coupler ce modèle à un modèle de stockage et de récolte

Utilisation du modèle pour d'autres applications:
- Multi-robots (Projet SYMBRION)
- Economie (Lopez)