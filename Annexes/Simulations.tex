
\begin{algorithm}
\caption{Simulations stochastiques synchrones d'une population homogène.}
\begin{algorithmic}

\State Nb$_{Simul} \gets 10000$
\State Nb$_{Indiv} \gets 1000$\\


\State $n \gets 100$
\State $Q_{tot} \gets 50$\\

\State ChargeUnit $\gets 1./n$\\

\State TF $\gets$ Table[$Nb_{Indiv}$][$Nb_{Simul}$]
\State \textcolor{blue}{On crée un tableau TF, t.q. TF[i][j] contient la charge de l'individu i au temps j.}\\
\State TF $\gets$ InitialiseTF($Q_{tot}$)
\State \textcolor{blue}{Initialise le tableau TF en distribuant la charge totale dans la population.}\\

\For{tps $\gets 0, Nb_{Simul}$}
	\State Indices $\gets$ shuffle(1,$Nb_{Indiv}$) 
	\State \textcolor{blue}{Crée une liste contenant les entiers de 1 à $Nb_{Indiv}$ mélangés aléatoirement.}\\
	\For{$i \gets 0, Nb_{Indiv}/2$}
		\State k $\gets$ Indices[2i]
		\State l $\gets$ Indices[2i+1] \\
		
		\State $x,y \gets random()$\\
		
		\State $l_1 \gets TF[k][tps]$
		\State $l_2 \gets TF[l][tps]$\\
		
		\If{$x<P_R(l_1)$ , $y > P_R(l_2) $}
			\State don $\gets$ ChargeUnit 
		\ElsIf{$x>P_R(l_1)$ , $y < P_R(l_2) $}
			\State don $\gets$ -ChargeUnit 
		\Else
			\State $don=0$
		\EndIf \\
		
		\State $TF[k][tps+1] \gets TF[k][tps]+don$
        \State $TF[l][tps+1] \gets TF[l][tps]-don$\\
	\EndFor
\EndFor



\end{algorithmic}
\end{algorithm}


\clearpage

\begin{algorithm}
\caption{Simulations stochastiques asynchrones des échanges d'une population homogène.}
\begin{algorithmic}

%\Function{evolution}{a}
%\State $a \gets a+1$
%\EndFunction
\State $Nb_{Simul} \gets 5 000 000$
\State $Nb_{Indiv} \gets 1000$\\

\State $n \gets 100$
\State $Q_{tot} \gets 50$\\

\State ChargeUnit $\gets 1./n$\\


\State TF $\gets$ Table[$Nb_{Indiv}$][$Nb_{Simul}$]
\State \textcolor{blue}{On crée un tableau TF, t.q. TF[i][j] contient la charge de l'individu i au temps j.}\\
\State TF $\gets$ InitialiseTF($Q_{tot}$)
\State \textcolor{blue}{Initialise le tableau TF en distribuant la charge totale dans la population.}\\

\For{tps $\gets 0, Nb_{Simul}$}
	
	\State k $\gets random(Nb_{Indiv})$
	\State l $\gets random(Nb_{Indiv})$
	\While{k==l}
		\State l $\gets random(Nb_{Indiv})$
	\EndWhile
	\State \textcolor{blue}{on assure $k \neq l$}\\
		
	\State $x,y \gets random()$\\
		
	\State $l_1 \gets TF[k][tps]$
	\State $l_2 \gets TF[l][tps]$\\
		
	\If{$x<P_R(l_1)$ , $y > P_R(l_2) $}
		\State don $\gets$ ChargeUnit 
	\ElsIf{$x>P_R(l_1)$ , $y < P_R(l_2) $}
		\State don $\gets$ -ChargeUnit 
	\Else
		\State $don=0$
	\EndIf \\
		
	\State $TF[k][tps+1] \gets TF[k][tps]+don$
    \State $TF[l][tps+1] \gets TF[l][tps]-don$\\
\EndFor



\end{algorithmic}
\end{algorithm}


Remarques:
\begin{itemize}
\item[$\bullet$]La quantité totale de nourriture dans la population notée $Q$ est définie par la charge moyenne des individus. Une quantité de nourriture $Q=25$ correspond à une charge moyenne par individu de 25 unités $\Delta l$.
\item[$\bullet$]La fonction InitialiseTF distribue la quantité totale de nourriture $Q$ au sein de la population pour le premier pas de temps.
\end{itemize}


\pagebreak

\begin{algorithm}
\caption{Simulations stochastiques des échanges sur une population hétérogène.}
\begin{algorithmic}

%\Function{evolution}{a}
%\State $a \gets a+1$
%\EndFunction
\State Nb$_{Simul} \gets 10000$
\State Nb$_{Indiv} \gets 1000$\\

\State n $\gets 100$
\State $Q_{tot} \gets 50$\\

\State ChargeUnit $\gets 1./n$\\


\State TF $\gets$ Table[$Nb_{Indiv}$][$Nb_{Simul}$]
\State \textcolor{blue}{On crée un tableau TF, t.q. TF[i][j] contient la charge de l'individu i au temps j.}\\
\State TF $\gets$ InitialiseTF($Q_{tot}$)
\State \textcolor{blue}{Initialise le tableau TF en distribuant la charge totale dans la population.}\\

\For{tps $\gets 0, Nb_{Simul}$}
	\State Indices $\gets$ shuffle(1,$Nb_{Indiv}$) 
	\State \textcolor{blue}{Crée une liste contenant les entiers de 1 à $Nb_{Indiv}$ mélangés aléatoirement.}\\
	\For{$i \gets 0, Nb_{Indiv}/2$}
		\State k $\gets$ Indices[2i]
		\State l $\gets$ Indices[2i+1] \\
		
		\State $x,y \gets random()$\\
		
		\State $l_1 \gets TF[k][tps]$
		\State $l_2 \gets TF[l][tps]$\\
		
		\If{$x<P_R(l_1,capacite(k))$ , $y > P_R(l_2,capacite(l)) $}
			\State don $\gets$ ChargeUnit 
		\ElsIf{$x>P_R(l_1,capacite(k))$ , $y < P_R(l_2,capacite(k)) $}
			\State don $\gets$ -ChargeUnit 
		\Else
			\State $don=0$
		\EndIf \\
		
		\State $TF[k][tps+1] \gets TF[k][tps]+don$
        \State $TF[l][tps+1] \gets TF[l][tps]-don$\\
	\EndFor
\EndFor



\end{algorithmic}
\end{algorithm}

Remarques:
\begin{itemize}
\item[$\bullet$]La quantité totale de nourriture dans la population notée $Q$ est définie par la charge moyenne des individus. Une quantité de nourriture $Q=25$ correspond à une charge moyenne par individu de 25 unités $\Delta l$.
\item[$\bullet$]Dans le tableau TF, les $\frac{N}{2}$ premiers individus sont considérés comme maigres, les $\frac{N}{2}$ suivants sont gros.
\item[$\bullet$]La fonction InitialiseTF distribue la quantité totale de nourriture $Q$ au sein de la population pour le premier pas de temps.
\item[$\bullet$]La fonction capacite(k) retourne $n$ si $0<k\leq\frac{N}{2}$, $2n$ si $\frac{N}{2}< k \leq N$ .
\end{itemize}



\pagebreak