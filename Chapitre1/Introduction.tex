\chapter{Introduction}

\section{Présentation et intérêt du travail}
I am sexy and I know it! à rédiger le jour J à 11h58

- slide p3
- continuité du travail de William

\section{Contexte biologique}

\subsection{Gestion de la nourriture}

Au sein de la colonie chaque fourmis occupe une fonction bien précise\footnote{Ces fonctions peuvent évoluer pour un individu, on constate (ref \fixme) que les rôles sont généralement attribués selon l'âge de l'individu.}, que ce soit la recherche de nourriture, son stockage, la ponte... La recherche et récolte de nourriture est assurée par les fourmis fourrageuses, le stockage par certaines domestiques, la redistribution par d'autres. Cette répartition des tâches, caractéristique des animaux sociaux, par rapport à la stratégie de récolte individuelle, libère du temps pour une partie de la population. Ce qui permet alors d'autres spécialisations tels la reproduction, la solidification du nid etc.\\


La gestion de la nourriture chez les fourmis peut être schématisée en 3 phases principales:

\begin{itemize}
\item[$\bullet$] Dans un premier temps, \textbf{la récolte de nourriture.}
Les 3 dynamiques: récolte (par cycle), échange (continu), stockage (continu)
\end{itemize}


\subsection{La dynamique d'échange}

Position du modèle, importance des différents facteurs à l'échelle de l'individu/de la population, explications sur la trophallaxie, ...

